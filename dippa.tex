\documentclass[a4paper]{article}

% Encoodaus, joka sopii suomenkielellä (esim. ä ja ö)
\usepackage[utf8]{inputenc}
\usepackage[T1]{fontenc}

% Suomenkielinen tavutus
\usepackage[finnish]{babel}

% Viitteet
\usepackage{natbib}

% Otsikkojen päätteetön fontti
\usepackage{sectsty}
\allsectionsfont{\sffamily\large}

% Viitteiden merkit
\bibpunct{(}{)}{;}{a}{,}{,}

\begin{document}

\title{\small T-76.5612 Software Project Management \\ Exercise 4, Pre-Lecture \\ \huge Hajautetun ohjelmistokehitysprojektin ongelmat}
\date{15.2.2012}
\author{Mikko Koski \\ mikko.koski@aalto.fi \\ 66467F}
\maketitle

\normalsize

\section{Johdanto}

Hajautetut ohjelmistoprojektit kohtaavat usein ongelmia, jotka saattavat hidastaa projektin etenemistä. Ongelmat voivat olla monisyisiä ja voivat johtua esimerkiksi johtamisesta, kommunikoinnista, osaamisen puutteesta tai kulttuurieroista. Erityisesti kommunikointiongelmat tuntuvat nousevan pinnalle lähes aina hajautettujen projektien ongelmista puhuttaessa.

\section{FinnSoft WHS -projektin suurimmat ongelmat}

FinnSoft WHS -projektissa esiintyi monia hyvin tyypillisiä hajautetun projektin ongelmia. Projektin suurimmat ongelmat liittyivät mielestäni kommunikointiin sekä tiimien välisen luottamuksen puuttumiseen. Projektin eri osapuolilla ei tuntunut olevan minkäänlaista tiimihenkeä. Kyseinen projekti on hyvä esimerkki siitä, että useinkaan pahimmat ongelmat eivät ole prosesseihin tai teknisiin haasteisiin liittyviä vaan pikemminkin ihmisiin ja tiimityöskentelyyn liittyviä ongelmia.

Alla on lueteltu projektin kahdeksan räikeintä ongelmaa tärkeysjärjestyksessä sekä selitetty lyhyesti miksi kyseinen ongelma on merkittävä.

% Tuotteen omistaja Ruotsissa
% Spesifikaattien kommunikointi toteuttajille
% Rahoitus kiinni toiminnallisuuksista
% Epäselvät roolit
% Ei yhtä sovittua palaveriaikaa
% Kokematonta työvoimaa
% Romania yrittää suojella itseään
% Erilaiset työkalut iteraation työtehtäville (Gemini ja Excel)
% Ei kunnollisia työkaluja (Romaniasta puuttuu lisenssejä)


\subsection{Tiimihengen ja luottamuksen puute}
% Uuden työvoiman lisääminen
% Työvoiman vähentäminen
% Liian vähän matkustelua
% Luottamuspula
% Tiedonkulku ei ole avointa
% Romanialla ei vaikutusmahdollisuuksia
% Raskas organisaatiorakenne

Projektin eri tiimit Suomessa ja Romaniassa eivät puhaltaneet samaan hiileen, vaikka he työskentelivätkin saman tuotteen parissa. Luottamuspula tiimien välillä heijastui tyytymättömyytenä sekä kommunikaatio-ongelmina. Projektitiimien jäsenet vierailivat toistensa luona liian vähän eikä yhteistä tiimihenkeä päässyt näin ollen syntymään.

Tiimihengen puute eri tiimien välillä ajoi projektin eri tiimit vahvoihin siiloihin. Pahimmillaan ongelmat muodostuivat niin pahoiksi, että esimerkiksi Romanian tiimi joutui suojaamaan itseään Suomen tiimiltä.

\subsection{Henkilöstön lisääminen ja vähentäminen}

Henkilöstön lisääminen projektiin nopeuttamisen toimesta on yksi klassisimmista virheistä. Monissa tutkimuksissa on todettu, että henkilöstön lisääminen projektiin pikemminkin hankaloittaa kuin nopeuttaa projektin edistymistä. Myös tässä projektissa projektin henkilöstö kasvoi huomattavasti, jonka seurauksena organisaatiosta tuli iso ja jäykkä. Myöhemmin säästöjen ja organisaation keventämisen toivossa työntekijöitä alettiin irtisanomaan.

Henkilöstön vaihtumisella on selkeät vaikutukset tiimihenkeen. On vaikea luoda yhtenäistä tiimiä jos henkilöstö vaihtuu jatkuvasti. 

\subsection{Kommunikaatio-ongelmat}
% Analyytikko ei toimi yhdessä ohjelmoijien kanssa
% Ei kunnollista englannin taitoa
% Ei videokonferensseja
% Kulttuuriset erot
% Asiakas-toimittaja kommunikointi vain suomeksi vain suomalaisten kesken

Luottamuksen puuttuessa myös tiimien välinen kommunikaatio hankaloitui. Tiimit keskustelivat keskenään chat-ohjelmien avulla, eivätkä esimerkiksi pitäneet videokonferensseja, vaikka videokonferenssien avulla on mahdollista saavuttaa paljon parempi läsnäolontunne, kuin puheluilla tai chat-ohjelmilla.

Myös kielelliset ja kulttuurilliset erot hankaloittivat kommunikaatiota, sillä osalla romanialaisista oli huono englannin kielen taito. On selvää, että kun suomalainen kääntää tietyn projektin vaatimuksen huonolla englannin kielellään suomesta englantiin, ja kun romanialainen lukee tämän vielä huonommalla englannin taidolla, on väärinymmärrysten mahdollisuus todella suuri.

\subsection{Tiedonkulkuongelmat}

Projektin sisällä tietoa ei jaettu tasapuolisesti eri tiimien välillä. Romanialaiset jäivät usein vaille ymmärrystä projektin ympäristöstä, koska suomalaiset kommunikoivat asiakkaalle suomeksi. Romanialaisille kommunikoitiin tämän jälkeen vain ne asiat, jotka koettiin välttämättömiksi.

Tiimien välinen luottamuspula näkyy tässäkin ongelmassa. Romanialaisia pidettiin projektissa vain työvoimana, eikä heille koettu tärkeäksi välittää kaikkea tietoa asiakkaalta. Kun romanialaisilta puuttui osa tiedosta, joka olisi auttanut heitä ymmärtämään paremmin ohjelmiston toimintaympäristöä heijastui se myös tuloksiin. 

\subsection{Epämotivoiva työnjako}

Romanialaiset kokivat itsensä vain työvoimaksi siinä missä projektin aivot olivat Suomessa. Romanialaisilla ei ollut mahdollista vaikuttaa projektin tai iteraatioiden suunnitteluun vaan nämä suunnitelmat tulivat määräyksenä Suomesta. Tällainen toimintatapa on eittämättä hyvin epämotivoivaa romanialaisten kannalta.

\subsection{Estimointi tehtiin muiden kuin toteuttajien taholta}
% Ei tehdä "reaalista" arviointia ja velocityyn perustuvaa ennakointia
% Estimointia tekivät henkilöt, jotka eivät vastanneet toteutuksesta
% Suunnitelmat tehdään ei-koodaajien toimesta

Projektin työtehtävien estimointi tehtiin Suomessa, kun taas itse toteutus tehtiin Romaniassa. Mielestäni tämä on täysin väärä menettelytapa. Työmääräarviointi pitäisi ehdottomasti tehdä toteuttavan tahon toimesta, sillä heillä on paras ja realistisin tieto siitä, kauanko tehtävän suorittamiseen menee. Jos työmääräarviointi olisi tehty romanialaisten toimesta, olisi heidän ollut myös paljon helpompi sitoutua pitämään kiinni antamastaan aikatauluarviosta.

\subsection{Epärealistiset tavoitteet}

Projektin yleinen aikataulu oli työntekijöiden mielestä epärealistinen. Epärealistinen aikataulu on omiaan vähentämään motivaatiota, jos tiimin jäsenet tietävät jo etukäteen, että tavoiteltu aikataulu on täysin mahdoton.

\subsection{Kasvava tekninen velka}
% Kasvanut tkninen velka
% Ei testattavaa
% Ei toimi "kerralla kuntoon" (monta testauskertaa)

Epärealistisista tavoitteista johtuen romanialaisilla ei ollut aikaa tehdä kunnollista laatua. Kovan paineen alla koodaustyötä tehtiin lyhytnäköisesti nopeasti kyhäämällä uudet toiminnallisuudet kokoon. Tästä syystä ohjelmiston kompleksisuus kasvoi ja samalla kasvoi myös ns. tekninen velka (engl. technical debt). Koska tätä velkaa ei missään vaiheessa maksettu takaisin korjaamalla aikaisemmat nopeat ratkaisut, hidasti tämä ohjelmiston kehitystä ohjelmiston koon kasvaessa.

\subsection{Iteratiivista prosessia ei toteutettu oikein}
% Liian pitkät iteraatiot alussa (6 kuukautta)
% Ei iteraatiodemoja
% Iteraatiot eivät merkitse mitään, koska niissä ei pysytä
% Aina roikkuvia taskeja edellisestä iteraatiosta

Projektin loppupuolella projektissa yritettiin käyttää iteratiivista prosessimallia, mutta sitä käytettiin hyödyksi todella heikosti. Iteraatioita ei suoritettu koskaan kerralla loppuun, vaan iteraatiosta jäi aina tehtäviä, jotka valuivat seuraavaan iteraatioon. Eräässä haastattelussa todettiin myös, että iteraatiot eivät merkinneet yhtään mitään vaikka ne olivatkin olemassa. Myöskään iteraatiodemoja ei pidetty.

Iteraatiot suunniteltiin aina Suomessa ja ne suunniteltiin epärealistisen isoiksi. Tämä virhe oltaisiin voitu helpostikin välttää antamalla romanialaisille enemmän vastuuta työtehtävien arvioinnista sekä seuraamalla toteutunutta nopeutta (engl. velocity). Tämän jälkeen seuraava iteraatio voitaisiin suunnitella realistisen kokoiseksi aikaisempien iteraatioiden nopeutta hyväksikäyttäen.

\end{document}