\documentclass[a4paper]{article}

% Encoodaus, joka sopii suomenkielellä (esim. ä ja ö)
\usepackage[utf8]{inputenc}
\usepackage[T1]{fontenc}

% Suomenkielinen tavutus
\usepackage[finnish]{babel}

% Viitteet
\usepackage{natbib}

% Otsikkojen päätteetön fontti
\usepackage{sectsty}
\allsectionsfont{\sffamily\large}

% Viitteiden merkit
\bibpunct{(}{)}{;}{a}{,}{,}

\begin{document}

\title{\small T-76.5612 Software Project Management \\ Pre-Lecture Exercise 3 \\ \huge Hajautetut ohjelmistokehitysprojektit}
\date{15.2.2012}
\author{Mikko Koski \\ mikko.koski@aalto.fi \\ 66467F}
\maketitle

\normalsize

\section{Johdanto}

Hajautetut ohjelmistoprojektit kohtaavat usein ongelmia, jotka saattavat hidastaa projektin etenemistä huomattavasti. Ongelmat voivat olla monisyisiä ja voivat johtua esimerkiksi johtamisesta, kommunikoinnista, osaamisen puutteesta tai kulttuurieroista. Erityisesti kommunikointiongelmat tuntuvat nousevan pinnalle hajautettujen projektien ongelmista puhuttaessa.

\section{FinnSoft WHS -projektin suurimmat ongelmat}

FinnSoft WHS -projektissa esiintyi monia hyvin tyypillisiä hajautetun projektin ongelmia. Projektin suurimmat ongelmat liittyivät mielestäni kommunikointiin sekä tiimien välisen luottamuksen puuttumiseen. Projektin osapuolilla ei tuntunut olevan minkäänlaista tiimihenkeä. Kyseinen projekti on hyvä esimerkki siitä, että useinkaan pahimmat ongelmat eivät ole prosesseihin tai teknisiin haasteisiin liittyviä vaan pikemminkin ihmisiin ja tiimityöskentelyyn liittyviä ongelmia.

Alla on lueteltu projektin kahdeksan räikeintä ongelmaa tärkeysjärjestyksessä sekä selitetty lyhyesti miksi kyseinen ongelma on merkittävä.

% Tuotteen omistaja Ruotsissa
% Spesifikaattien kommunikointi toteuttajille
% Rahoitus kiinni toiminnallisuuksista
% Epäselvät roolit
% Ei yhtä sovittua palaveriaikaa
% Kokematonta työvoimaa
% Romania yrittää suojella itseään
% Erilaiset työkalut iteraation työtehtäville (Gemini ja Excel)
% Ei kunnollisia työkaluja (Romaniasta puuttuu lisenssejä)


\subsection{Tiimihengen ja luottamuksen puute}
% Uuden työvoiman lisääminen
% Työvoiman vähentäminen
% Liian vähän matkustelua
% Luottamuspula
% Tiedonkulku ei ole avointa
% Romanialla ei vaikutusmahdollisuuksia
% Raskas organisaatiorakenne

Projektin eri tiimit Suomessa ja Romaniassa eivät tunteneet olevansa samaa porukkaa, vaikka he työskentelivätkin saman tuotteen parissa. Luottamuspula tiimien välillä heijastui tyytymättömyytenä sekä kommunikaatio-ongelmina. Projektitiimien jäsenet vierailivat toistensa luona liian vähän eikä yhteistä tiimihenkeä päässyt näin ollen syntymään. 

\subsection{Kommunikaatio-ongelmat}
% Analyytikko ei toimi yhdessä ohjelmoijien kanssa
% Ei kunnollista englannin taitoa
% Ei videokonferensseja
% Kulttuuriset erot
% Asiakas-toimittaja kommunikointi vain suomeksi vain suomalaisten kesken

Luottamuksen puuttuessa myös tiimien välinen kommunikaatio hankaloitui. Tiimit keskustelivat keskenään chat-ohjelmien avulla, eivätkä esimerkiksi pitäneet videokonferensseja, vaikka videokonferenssien avulla on mahdollista saavuttaa paljon parempi läsnäolontunne, kuin puheluilla tai chat-ohjelmilla.

\subsection{Tiedonkulkuongelmat}

Projektin sisällä tietoa ei jaettu tasapuolisesti eri tiimien välillä. Romanialaiset jäivät usein vaille ymmärrystä projektin ympäristöstä, koska suomalaiset kommunikoivat asiakkaalle suomeksi. Romanialaisille kommunikoitiin tämän jälkeen vain ne asiat, jotka koettiin välttämättömiksi.

\subsection{Epämotivoiva työnjako}

Romanialaiset kokivat itsensä vain työvoimaksi siinä missä projektin aivot olivat Suomessa. Romanialaisilla ei ollut mahdollista vaikuttaa projektin tai iteraatioiden suunnitteluun vaan nämä suunnitelmat tulivat määräyksenä Suomesta. Tällainen toimintatapa on eittämättä hyvin epämotivoivaa romanialaisten kannalta.

\subsection{Estimointi tehtiin muiden kuin toteuttajien taholta}
% Ei tehdä "reaalista" arviointia ja velocityyn perustuvaa ennakointia
% Estimointia tekivät henkilöt, jotka eivät vastanneet toteutuksesta
% Suunnitelmat tehdään ei-koodaajien toimesta

Projektin työtehtävien estimointi tehtiin Suomessa, kun taas itse toteutus tehtiin Romaniassa. Mielestäni tämä on täysin väärä menettelytapa. Työmääräarviointi pitäisi ehdottomasti tehdä toteuttavan tahon toimesta, sillä heillä on paras ja realistisin tieto siitä, kauanko tehtävän suorittamiseen menee. Jos työmääräarviointi olisi tehty romanialaisten toimesta, olisi heidän ollut myös paljon helpompi sitoutua siihen.

\subsection{Epärealistiset tavoitteet}

Projektin yleinen aikataulu oli työntekijöiden mielestä epärealistinen. Epärealistinen aikataulu on omiaan vähentämään motivaatiota, jos tiimin jäsenet tietävät jo etukäteen, että tavoiteltu aikataulu on täysin mahdoton.



\subsection{Kasvava tekninen velka}
% Kasvanut tkninen velka
% Ei testattavaa
% Ei toimi "kerralla kuntoon" (monta testauskertaa)

\subsection{Iteratiivista prosessia ei toteuteta oikein}
% Liian pitkät iteraatiot alussa (6 kuukautta)
% Ei iteraatiodemoja
% Iteraatiot eivät merkitse mitään, koska niissä ei pysytä
% Aina roikkuvia taskeja edellisestä iteraatiosta

\section{Yhteenveto}

\end{document}