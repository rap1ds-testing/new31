\documentclass[a4paper]{article}

% Encoodaus, joka sopii suomenkielellä (esim. ä ja ö)
\usepackage[utf8]{inputenc}
\usepackage[T1]{fontenc}

% Suomenkielinen tavutus
\usepackage[finnish]{babel}

% Viitteet
\usepackage{natbib}

% Otsikkojen päätteetön fontti
\usepackage{sectsty}
\allsectionsfont{\sffamily\large}

% Viitteiden merkit
\bibpunct{(}{)}{;}{a}{,}{,}

\begin{document}

\title{\small T-76.5612 Software Project Management \\ Pre-Lecture Exercise 3 \\ \huge Hajautetut ohjelmistokehitysprojektit}
\date{15.2.2012}
\author{Mikko Koski \\ mikko.koski@aalto.fi \\ 66467F}
\maketitle

\normalsize

\section{Johdanto}

\section{Projektin suurimmat ongelmat}

% Uuden työvoiman lisääminen
% Työvoiman vähentäminen
% Tuotteen omistaja Ruotsissa
% Liian vähän matkustelua
% Asiakas-toimittaja kommunikointi vain suomeksi vain suomalaisten kesken
% Liian pitkät iteraatiot alussa (6 kuukautta)
% Estimointia tekivät henkilöt, jotka eivät vastanneet toteutuksesta
% Kommunikaatio-ongelmat
% Raskas organisaatiorakenne
% Spesifikaattien kommunikointi toteuttajille
% Rahoitus kiinni toiminnallisuuksista
% Luottamuspula
% Epäselvät roolit
% Mahdoton deadline yleisellä tasolla
% Ei yhtä sovittua palaveriaikaa
% Ei iteraatiodemoja
% Iteraatiot eivät merkitse mitään, koska niissä ei pysytä
% Tiedonkulku ei ole avointa
% Ei toimi "kerralla kuntoon" (monta testauskertaa)
% Kulttuuriset erot
% Kokematonta työvoimaa
% Romania yrittää suojella itseään
% Analyytikko ei toimi yhdessä ohjelmoijien kanssa
% Ei videokonferensseja
% Aina roikkuvia taskeja edellisestä iteraatiosta
% Ei tehdä "reaalista" arviointia ja velocityyn perustuvaa ennakointia
% Erilaiset työkalut iteraation työtehtäville (Gemini ja Excel)
% Kasvanut tekninen velka
% Ei testattavaa
% Suunnitelmat tehdään ei-koodaajien toimesta
% Romanialla ei vaikutusmahdollisuuksia
% Ei kunnollisia työkaluja (Romaniasta puuttuu lisenssejä)
% Ei kunnollista englannin taitoa

\section{Yhteenveto}

\end{document}