\documentclass[a4paper]{article}

% Encoodaus, joka sopii suomenkielellä (esim. ä ja ö)
\usepackage[utf8]{inputenc}
\usepackage[T1]{fontenc}

% Suomenkielinen tavutus
\usepackage[finnish]{babel}

% Viitteet
\usepackage{natbib}

% Otsikkojen päätteetön fontti
\usepackage{sectsty}
\allsectionsfont{\sffamily\large}

% Viitteiden merkit
\bibpunct{(}{)}{;}{a}{,}{,}

\begin{document}

\title{\small T-76.5612 Software Project Management \\ Pre-Lecture Exercise 3 \\ \huge Hajautetut ohjelmistokehitysprojektit}
\date{15.2.2012}
\author{Mikko Koski \\ mikko.koski@aalto.fi \\ 66467F}
\maketitle

\normalsize

\section{Johdanto}

Hajautetut ohjelmistoprojektit kohtaavat usein ongelmia, jotka saattavat hidastaa projektin etenemistä huomattavasti. Ongelmat voivat olla monisyisiä ja voivat johtua esimerkiksi johtamisesta, kommunikoinnista, osaamisen puutteesta tai kulttuurieroista. Erityisesti kommunikointiongelmat tuntuvat nousevan pinnalle hajautettujen projektien ongelmista puhuttaessa.

\section{FinnSoft WHS -projektin suurimmat ongelmat}



% Tuotteen omistaja Ruotsissa
% Spesifikaattien kommunikointi toteuttajille
% Rahoitus kiinni toiminnallisuuksista
% Epäselvät roolit
% Ei yhtä sovittua palaveriaikaa
% Kokematonta työvoimaa
% Romania yrittää suojella itseään
% Erilaiset työkalut iteraation työtehtäville (Gemini ja Excel)
% Ei kunnollisia työkaluja (Romaniasta puuttuu lisenssejä)


\subsection{Tiimihengen ja luottamuksen puute}
% Uuden työvoiman lisääminen
% Työvoiman vähentäminen
% Liian vähän matkustelua
% Luottamuspula
% Tiedonkulku ei ole avointa
% Romanialla ei vaikutusmahdollisuuksia
% Raskas organisaatiorakenne

\subsection{Kommunikaatio-ongelmat}
% Analyytikko ei toimi yhdessä ohjelmoijien kanssa
% Ei kunnollista englannin taitoa
% Ei videokonferensseja
% Kulttuuriset erot
% Asiakas-toimittaja kommunikointi vain suomeksi vain suomalaisten kesken

\subsection{Tiedonkulkuongelmat}

\subsection{Epämotivoiva työnjako}

\subsection{Estimointi tehtiin muiden kuin toteuttajien taholta}
% Ei tehdä "reaalista" arviointia ja velocityyn perustuvaa ennakointia
% Estimointia tekivät henkilöt, jotka eivät vastanneet toteutuksesta
% Suunnitelmat tehdään ei-koodaajien toimesta

\subsection{Epärealistiset tavoitteet}
% Mahdoton deadline yleisellä tasolla

\subsection{Kasvava tekninen velka}
% Kasvanut tkninen velka
% Ei testattavaa
% Ei toimi "kerralla kuntoon" (monta testauskertaa)

\subsection{Iteratiivista prosessia ei toteuteta oikein}
% Liian pitkät iteraatiot alussa (6 kuukautta)
% Ei iteraatiodemoja
% Iteraatiot eivät merkitse mitään, koska niissä ei pysytä
% Aina roikkuvia taskeja edellisestä iteraatiosta

\section{Yhteenveto}

\end{document}